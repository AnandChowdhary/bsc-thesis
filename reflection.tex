\documentclass{article}

\title{Reflection Report}
\date{2020-06-06}
\author{Anand Chowdhary (s1930702)}

\begin{document}
  \pagenumbering{gobble}
  \maketitle
  \newpage
  \pagenumbering{arabic}

\section{Introduction}

Last year, over 100 billion emails were sent every day (Email Statistics Report, 2015-2019). A very common use case of sending emails in a work environment is to set up appointments for in-person or virtual meetings. However, several email exchanges are required in order to find a suitable time and place where all parties are available, and this back-and-forth calendar conflict resolution wastes 17 minutes per meeting on average (Mortensen, 2017). This can add up to a significant waste of time (over 100 hours per year), which can be otherwise used for productive work.

For some professionals, scheduling these meetings manually can feel like a "frustrating distraction from the things that matter" (J. Cranshaw et al., 2017) so much so that they hire assistants to help with the task. However, not everyone can afford full-time assistants and will therefore turn to software solutions. Intelligent virtual assistants over email built using machine learning can help by automating these scheduling messages. An AI assistant can access a user’s calendar and can find empty meeting slots based on the location of the user location and their scheduling preferences, and then send and respond to emails on their behalf.

In this graduation project, I aim to design and develop an AI-powered intelligent virtual assistant that automates the email scheduling process. Professionals will be able to use the web interface of the assistant service to solve the "time waste" problem when it comes to appointment scheduling. In the future, the capabilities of the assistant can be extended to essentially everything a human assistant can do, from sending outbound marketing emails and following up with coworkers on tasks, to automating other parts of the professional’s life. Of course, the product also has several ethical implications and possible social disruptions, such as job loss for secretaries by encouraging automation and ensuring data and privacy protection, apart from answering the main ethical conundrum — whether end users who receive emails from the service are informed that the email is written by an AI assistant, not a real human.


\subsection{About EIVA}

Structuring a document is easy!

\subsection{Research Methodology}

Structuring a document is easy!

\section{Ethical Implications}

OK byte

\subsection{Job loss due to automation}

Just like in other applications of automation software, loss of employment for assistants is an important social disruption that this product unfortunately encourages. In the interest of saving both time and money, companies may choose to deploy AI-powered assistants on an organization-wide level and terminate the employment of all their secretaries.

\subsection{``Sent by an AI"}

If you read major American newspapers such as Forbes or the Los Angeles Times, chances are you’ve already read a story entirely written by an AI-powered software system. This process is known as automated journalism, and highlights an important question about authorship. A study found that participants attribute story credit to the programmers who developed the AI or the news organization publishing the story (Montal, Reich, 2005).

More interestingly, there is no visible indicator for readers to verify whether an article was written by a robot or human, which raises issues of transparency (Dörr, Hollnbuchner, 2017). Similarly, if an AI assistant is impersonating a human assistant to send emails on the professional's behalf, it raises the same ethical question of whether the end user receiving the email should know that it was not written by a human.

\end{document}
